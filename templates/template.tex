
% ----------------------------
%   字型設定
% ----------------------------

\usepackage{xeCJK}
\usepackage{CJKnumb}    
\usepackage{zhnumber}   %提供了用于排版中文数字表示的命令
\usepackage{amsfonts}   %用於數學的一組擴展字體
% \usepackage{CJKutf8}


% setup font
\defaultfontfeatures{AutoFakeBold=2.5,AutoFakeSlant=.2}
\setmainfont[
  Path={./fonts/},
  BoldFont={timesbd},
  ItalicFont={timesi},
  BoldItalicFont={timesbi}
]{times}   % 英文字體

\setCJKmainfont[
  Path=./fonts/ , 
  % AutoFakeBold=true,
  % AutoFakeSlant=true
]{\CJKmainfont}       %中文字體

\setCJKfamilyfont{kai}[Path=./fonts/]{kaiu}
\newcommand{\fontkai}{\CJKfamily{kai}} %標楷體
\setCJKfamilyfont{ming}[Path=./fonts/,BoldFont={mingliub}]{mingliu}
\newcommand{\fontming}{\CJKfamily{ming}} %新細明體

% \linespread{1.5}


% ----------------------------
%   超連結
% ----------------------------
\usepackage{hyperref}
\usepackage{url}

\AtBeginDocument{
  \hypersetup{
    pdfpagemode={UseOutlines},
    linktoc=all,
    unicode,
    hidelinks,
    pdfcreator={\@authoren \ \@authorzh},
	  pdfproducer={\@authoren  , \@advisoren},
    pdftitle={\@titleen},
    pdfauthor={\@authoren},
    pdfsubject={\@typeen{} \@classen},
    pdfkeywords={\@keywordsen}
  }
}


% ----------------------------
%   浮水印
% ----------------------------

\usepackage[contents={}]{background} %如果沒有給 contest空白會自動加上 draft 浮水印字樣

% 浮水印參數定義
\newcommand\watermark {
	\ifdefined\withwatermark
    \backgroundsetup{
      contents={\includegraphics[]{\watermarkimage}},
      scale=1.25,
      opacity=0.2,
      angle=0
    }
    \fi
}


% ----------------------------
%   標題
% ----------------------------

\usepackage{subcaption}
\usepackage{caption}
\usepackage[center]{titlesec}
\titleformat{\chapter}{\centering\Huge\bfseries}{第\,\CJKnumber\thechapter\,章}{1em}{}
\titleformat{\section}{\raggedright\Large\bfseries}{\,\thesection\,}{1em}{}
\titleformat{\subsection}{\raggedright\large\bfseries}{\,\thesubsection\,}{1em}{}
\titlespacing*{\chapter}{0pt}{0pt}{40pt}
\captionsetup{labelsep=quad} % graphics caption ':' -> ' '


% ----------------------------
%   目錄
% ----------------------------

\usepackage{titletoc}
\usepackage{xpatch}
\titlecontents{chapter}[0em]{}{第\CJKnumber{\thecontentslabel}章\quad}{}{\titlerule*[1em]{}\contentspage}
\renewcommand{\contentsname}{目錄}
\renewcommand{\tablename}{表}
\renewcommand{\listtablename}{表目錄}
\renewcommand{\figurename}{圖}
\renewcommand{\listfigurename}{圖目錄}


% ----------------------------
%   排版擴充
% ----------------------------

\usepackage{indentfirst}  %首行縮排
\usepackage{setspace} % 行距
\usepackage[section]{placeins}
\usepackage{templates/template_layout}
% \usepackage{fancyhdr}

% prevent placing floats before a section
\let\Oldsubsection\subsection
\renewcommand{\subsection}{\FloatBarrier\Oldsubsection}

% \setlength{\parindent}{0.88cm} %縮排設定
\setlength{\parindent}{2em} %縮排設定


% ----------------------------
%   數學
% ----------------------------

\usepackage{amsmath}
\usepackage{amssymb}
\usepackage{amsthm}
\usepackage{siunitx}
\usepackage{array}
\usepackage{dsfont}
\usepackage{mathtools}


% ----------------------------
%   繪圖
% ----------------------------

\usepackage{graphicx}
\usepackage{tikz}
\usepackage{pgfplots}
\usepackage[subpreambles]{standalone}
\usepackage{amscd}

\graphicspath{ {./figures/} }
\pgfplotsset{compat=1.16}

\usetikzlibrary{calc,matrix,decorations.markings,decorations.pathreplacing,decorations.text}
\usetikzlibrary{arrows,shapes,chains}
\usetikzlibrary{automata,arrows.meta,positioning} % draw states


% ----------------------------
%   表格
% ----------------------------

\usepackage{multirow}
\usepackage{longtable}


% ----------------------------
%   程式碼
% ----------------------------

\usepackage[nounderscore]{syntax}
\usepackage{listings}

% source code highlighting
\lstset{
  numbers=none,
  captionpos=b,
  tabsize=2,
  basicstyle=\small,
  frame = single
}


% ----------------------------
%   演算法
% ----------------------------

% \usepackage{algorithm}
% \usepackage[linesnumbered,lined,boxed,commentsnumbered]{algorithm2e}
\usepackage[linesnumbered,noline,boxed]{algorithm2e}
% \usepackage[noend]{algpseudocode}
\usepackage{algpseudocode}
\usepackage[skins]{tcolorbox}
\usepackage{float}

% ----------------------------
%   條列式清單
% ----------------------------

\usepackage{enumerate}


% ----------------------------
%   額外擴充
% ----------------------------

\usepackage{ulem} %下劃線,刪除線.....






% bibliography
% \usepackage[redeflists]{IEEEtrantools}
\usepackage[backend=biber,sorting=none,style=ieee,seconds=true]{biblatex}
\addbibresource{reference.bib}



